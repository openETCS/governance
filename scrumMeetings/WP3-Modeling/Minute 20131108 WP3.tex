
\documentclass[a4paper, 11pt]{article}
\usepackage[ascii]{inputenc}
\usepackage{supertabular}
\usepackage[ngerman]{babel}
\usepackage{amsmath}
\usepackage{amssymb,amsfonts,textcomp}
\usepackage {geometry}
\geometry{a4paper,top=25mm,left=30mm,right=25mm,bottom=30mm}
\usepackage{color}
\usepackage{array}
\usepackage{hhline}
\usepackage{hyperref}
\hypersetup{colorlinks=true, linkcolor=blue, citecolor=blue, filecolor=blue, urlcolor=blue}


\begin{document}
{\begin{center}\huge\bf openETCS WP3 Meeting Minutes\end{center}}
\section{Meeting Information}

\renewcommand{\arraystretch}{1.5}
\begin{supertabular}{m{.2\textwidth}m{.8\textwidth}}
%\hline
Subject & WP3 Weekly Scrum: Modelling\\
Date \& time & 2013-11-08, 09:00h--10:00h\\
Location & Telco and Goto-Meeting\\
Called up by & Pierre Francois Jauquet\\
Participants &
Stephane Besure,
Nicolas Boverie,
Christian Giraud,
%Guillaume Durand,
%Yoann Guyot,
%Baseliyos Jacob,
%Klaus-R\"udiger Hase,
Bernd Hekele,
Pierre Francois Jauquet,
Vincent Nuhaan,
Marielle Petit-Doche,
%Uwe Steinke,
Jan Welte,
Jan Welvaarts
%Niklas Schaffrath
\\

Minutes by & Bernd Hekele\\
%\hline
\end{supertabular}
\renewcommand{\arraystretch}{1.0}

%\line(1,0){440}

\section{{Agenda}}
\begin{itemize}
\item Team 1 Progress
\item Team 2 Progress
\item Team 3 Progress
\item Team 4 Progress
\item Data Dictionary Team Progress
\item Organisational Issues 
\end{itemize}

\section{Discussion}
\begin{itemize}
\item Note: Location of Meeting Minutes\\
You can find these minutes here: \url{https://github.com/openETCS/governance/blob/master/scrumMeetings/WP3-Modeling/Minute%2020131108%20WP3.pdf}. Please, use the issue tracker for findings in the minutes.
 
\item All the teams reported progress. For all teams the next result will be made available on the repository today or latest at Tuesday next week. Also draft results are welcome. For those not being familiar with Github, mailing of information is accepted (send to Bernd Hekele).

\item The Data Dictionary team will analyse the results and report during next week's meeting  on the problems with the chosen procedure.

The Data Dictionary-Team has defined a temporary procedure on how to handle Data Dictionary requirements. This information is available in the SRS-Repository Wiki: \url{https://github.com/openETCS/SRS-Analysis/wiki/Data-Dictionary}.

The Data Dictionary-Team will collect an initial list of priority common data as a starting point.

The Data Dictionary users first have to check in the list of available data if there needs are already satisfied. If no data exist, the new requirement can be posted.

The template has been updated. The split in two files is proposed in order to take care on the different nature of the information collected in the templates. The main part covering the formal section of the old template will become part of an spreadsheet. The informal textual part can be linked to this spreadsheet as a word document.
\end{itemize}

\begin{enumerate}

\item Meeting Organisation\\
The weekly telco will be scheduled for Friday's, 9:00 - 10:00. The meeting will make use of the "weekly scrum goto-meeting" which is open for other participants of openETCS.

\item Workshop Organisation\\
The next workshop is scheduled for November 26th - November 28th.
Location is in clarification. Baseline is Alstom as Brussels. If Siemens want to provide a  location in Berlin by mid of next week this will be taken as preferred location.
Note: Ralf Pinger (Siemens) has been informed about this issue and promised to give feedback as soon as possible.

\item Effort Estimation\\
Pierre-Francois effort estimation for the task is no in internal review at Alstom. The result is planned to be made available to Siemens at beginning of next week. A feedback on the estimation can be discussed in next weeks meeting.

\end{enumerate}


\section{Notes}

\end{document}
