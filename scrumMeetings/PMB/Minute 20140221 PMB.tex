
\documentclass[a4paper, 11pt]{article}
\usepackage[ascii]{inputenc}
\usepackage{supertabular}
\usepackage[ngerman]{babel}
\usepackage{amsmath}
\usepackage{amssymb,amsfonts,textcomp}
\usepackage {geometry}
\geometry{a4paper,top=25mm,left=30mm,right=25mm,bottom=30mm}
\usepackage{color}
\usepackage{array}
\usepackage{hhline}
\usepackage{hyperref}
\hypersetup{colorlinks=true, linkcolor=blue, citecolor=blue, filecolor=blue, urlcolor=blue}

\begin{document}
{\begin{center}\huge\bf openETCS PMB Meeting Minutes\end{center}}
%{\begin{center}\huge\bf Project Management Board (PMB)\end{center}}
\section{Meeting Information}

\renewcommand{\arraystretch}{1.5}
\begin{supertabular}{m{.2\textwidth}m{.8\textwidth}}
%\hline
Subject & PMB Weekly Scrum: Project Management Board\\
Date \& time & 2014-02-21, 11:30h--12:00h\\
Location & Telco and Goto-Meeting\\
Called up by & Bernd Hekele\\
%Called up by & Klaus-R\"udiger Hase\\
Participants &
%Sylvain Baro,
Marc Behrens,
%Benjamin Beichler,
%Cyril Cornu,
%Klaus-R\"udiger Hase,
Bernd Hekele,
%Baseliyos Jacob,
%Gilles Dalmas,
%Pierre Francois Jauquet,
%Ainhoa Gracia,
%Frank Golatowski,
%Jonas Helming,
Michael Jastram,
%Pierre-Francois Jauquet,
%Roberto Kretschmer,
%Bego\~na Laibarra,
Peter Mahlmann,
Alexander Nitsch,
Matthieu Perin,
Marielle Petit-Doche,
%Stan Pinte,
Stefan Rieger,
%Alexander Stante,
Uwe Steinke
Izaskun de la Torre,
Jan Welte
%Jan Welvaarts
Giovanni Zanelli
\\

Minutes by & Bernd Hekele\\

%\hline
\end{supertabular}
\renewcommand{\arraystretch}{1.0}

%\line(1,0){440}

\section{{Agenda}}
%You can find a regular update on this location: \url{https://github.com/openETCS/governance/wiki/Project-Management}

\begin{itemize}
\item Progress on Itea Report
\item Use of Tools for Modelling
\item V\&V Internal Assessment workshop in N\"urnberg.\\

\end{itemize}

\section{Discussion}
\begin{itemize}
\item Progress on Itea Report
The report is in the status of a final review and will be submitted toItea on schedule.

\item V\&V Internal Assessment workshop in N\"urnberg.\\
The internal assessment workshop has been during this week. 
The reviewers (All4Tec and AEbt) concentrated on the process docuements and the quality assurance plan. During the meeting several issues were highlighted which needs completion before continuation of the internal assessment can be  performed.

To continue the work on the QA-Plan the QA backlog will be updated with findings from the meeting. A close follow-up of items in the backlog is planned to be part of the weekly QA session. (Thursday's 13:00 - 14:00) Active contribution by partners of WP1, WP2, WP3 and WP4 is needed to complete the task.

The next step of the internal assessment is planned to take place at June 10th (as a part of the Munich meeting).

A go/nogo decision  is scheduled for May 9th (PMB).

The meeting minutes and slides will be made available  in the V\&V project.

\item Use of Tools for Modelling\\
WP3 decided (already some time ago) to start the modelling activities with SCADE System (SysML) and SCADE Suite. Papyrus can be taken as an alternative to SCADE System if the restrictions documented in the modelling guideline (D2.4) are considered.

This decision is valid for Modelling activities. It is not a limitation for work in other WPs like WP4 or WP5.

An outcome of the Internal Assessment was the need to document all tool decisions on the development branch in the QA guide. The implementation of this proposal will be taken into the QA meeting.

\end{itemize}

\section{Notes}

\end{document}
